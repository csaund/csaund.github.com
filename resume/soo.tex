
\documentclass{article}
\usepackage[T1]{fontenc}
\usepackage[utf8]{inputenc}
\usepackage[margin=1in]{geometry}


\newcommand{\HRule}{\rule{\linewidth}{0.5mm}}
\newcommand{\Hrule}{\rule{\linewidth}{0.3mm}}

\makeatletter% since there's an at-sign (@) in the command name
\renewcommand{\@maketitle}{%
  \parindent=0pt% don't indent paragraphs in the title block
  \centering
  {\Large \bfseries\textsc{\@title}}
  \HRule\par%
  \textit{\@author \hfill \@date}
  \par
}
\makeatother% resets the meaning of the at-sign (@)

\title{Statement of Objectives}
\author{Faculty of Interest: Rosalind Picard, Deb Roy, Cynthia Breazeal \\ Carolyn Saund}
\date{Ph.D. Applicant} 

\setlength{\parindent}{4em}

\begin{document}
  \maketitle% prints the title block
  \thispagestyle{empty}
  \vspace{16pt}

	Empathetic and emotional artificial intelligence is an area of research with incredible potential to positively impact the lives of many people. With this as my motivation, I want to build an abstract emotional framework that allows artificially intelligent agents to collect valuable information about their human environment, which in turn allows them to better address those humans' emotional needs. I intend to pursue graduate studies to develop the academic ideas that will serve as a foundation for me to impact meaningful human communication on a global scale, and then become an industry leader in the field of Emotional AI by bringing those ideas to fruition. I work for two start ups born from the Media Lab, and observe many more continue to shape the limits of our interactions with technology through innovative, cross-discipline ideas that are broadly applicable, and put into action. The Media Lab will maximize the potential of my work to positively impact empathetic human connection on a global scale. \par
	My passions for empathetic communication and urge to solve problems are reflected in my academic and professional history. In my senior project I algorithmically implemented human-like category formation, which cemented my interest in cognitive architecture, and also illustrated the challenge of employing over-specialized or single-purpose algorithms in broad cognitive contexts. I want to contribute to algorithmic pieces that assemble to create a generally artificial intelligence, in order to ensure the first generally intelligent artificial being has the core trait of empathy. \par
	My career began at Jibo, a robotics company co-founded by Cynthia Breazeal. As a developer on the Software Developer Kit team, I learned to design easily understandable interfaces by implementing tools to program our bot for both internal and external contributors. Inspired by the creative potential of the nascent field of social robotics, I transferred to the Character AI team, where I implemented behaviors specifically designed to make Jibo feel more alive -- though not necessarily human! There, I explored literature on artificial emotional systems to contribute largely to a bi-axial emotional system, and to the infrastructure which allows Jibo to maintain a personal, internal state to vary his responses to environmental stimuli. This taught me how to quickly prototype and evaluate different system architectures to balance theoretical impact with production and resource constraints. I left my position at Jibo earlier this year, focused on research fundamentals behind what makes people connect with our robot. I am thrilled to see Jibo delight consumers in thousands of homes, because my work and ideas visibly transform interactions with technology to make them seamless, personalized, and compassionate. I eagerly anticipate continuing to advance concepts that have potential to benefit human connections to technology, and to one another in graduate school. \par
	Four months ago I moved to Cogito, co-founded by Sandy Pentland. Here, we pursue empathetic communication by coaching human conversation via empathetic cues. As a Senior Developer I design low-level interfaces through a holistic, user-centric lens, and implement full-stack development of these features. I also orchestrate collaboration between four engineering teams to design and implement data architecture of large systems, and have redesigned our release and communication processes to better facilitate cooperation among engineers. Through these experiences, I learned how to effectively structure and manage large technical projects and teams. I also learned how to structure experiences and products around the humans using them. Even with this full-time position, I continue to consult at Jibo to fulfill my desire to play with empathy in artificially intelligent systems, and look forward to working in a fully research-driven context. \par
	I emphatically believe seeking local impact is the strongest mechanism to foster connection across communities and build empathy within myself firsthand. I volunteer up to 100 hours per year with STEM initiatives such as Girls Who Code, Cambridge Inventors Club, and FIRST Robotics, where I practice teaching technical topics and engineering first principles to children with no prior exposure. I am the inaugural recipient of the Culture Award at Jibo, which recognizes outstanding cultural contribution. There, I also founded the diversity interest group, and at both organizations I lead groups and efforts to reconstruct our recruitment processes to address difficulties of encouraging applicants from minority demographics. I look forward to new challenges as a cultural, community, and technical leader as a graduate student. \par
	The research conducted at the Media Lab perfectly aligns with my academic interests in bettering communication, and my aspirations to build tools to do so.  The Affective Computing group, run by Rosalind Picard, is at the forefront of measuring and understanding emotive reactions, employing biophysical sensors to study emotional reactions in autistic children in the EngageMe project. Her approach could potentially be applied to all manner of social dynamics, from care-giving to friendship-building, to determine just how empathetic communication mentally impacts us. This is exhilarating to me, as this foundation of understanding empathetic human interaction can be used to develop strategies to improve connections across all human communities. Deb Roy of Social Machines spearheads interdisciplinary media projects that cross machine learning with social media in applications such as in rumor detection to analyze the spread of information and human relationships, which digs into the heart of what makes people feel connected to information, and to each other. And the Personal Robotics group, led by Cynthia Breazeal, makes learning more personalized and accessible to children by employing social communication skills. All of these projects are about improving meaningful connection between humans and the pervasive technologies of current society, which coincides with my goal of globally increasing empathetic communication, and consequently improving well-being. \par
	The combination of affective computing, social dynamics, and robotic interfaces will undoubtedly revolutionize human relationships with technology. Developing emotionally aware agents who behave empathetically, and folding them into technology to broadly impact society, is a task about which I am deeply passionate, and for which I am well prepared through experience working on an emotional robot and leading technical projects. The Media Lab is at the intersection of these technological domains, where innovation flourishes into tangible impact. By applying my skills to this problem, I hope to push the frontiers of interdisciplinary technology and lead the way finding solutions to facilitate effective and applicable empathetic communication between humans and machines.
\end{document}





